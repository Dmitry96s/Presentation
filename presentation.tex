% https://habrahabr.ru/post/145523/
% http://fsweb.info/editors/latex/presentation.html
\documentclass[10pt]{beamer}
\usepackage[utf8]{inputenc}
\usepackage[T2A]{fontenc}
\usepackage[english, russian]{babel}
\usepackage{amsmath,amsfonts,amssymb}
\usepackage{graphicx}
%\usetheme{Singapore}
% \usetheme{Berlin}
% \usetheme{CambridgeUS}
\usetheme{CambridgeUS}

%\setbeamercolor{элемент}{bg=цвет1,fg=цвет2}. Здесь «элемент» — название элемента, чей цвет мы хотим изменить (например, «normal text» — обычный текст), «bg» — цвет фона, «fg» — цвет текста.
%\usebackgroundtemplate{\includegraphics[width=\paperwidth,height=\paperheight]{my_backgroung_picture.jpg}}

\addto\captionsrussian{\renewcommand{\figurename}{Рис.}}
\setbeamertemplate{caption}[numbered]

\RequirePackage{caption}
\DeclareCaptionLabelSeparator{defffis}{ -- }
\captionsetup{justification=centering,labelsep=defffis}

\newcommand{\MB}{\mathbf}
\newcounter{myexmpl}



\begin{document}


\title{Эргономика компьютерных игр}
\subtitle{презентация по курсу ``Человеко-компьютерное взаимодействие''}
\logo{\includegraphics[width=0.15\textwidth]{res/img/titlePage/sfedu.png}}
\author{В.Э. Смирнов,
        Д.В. Сухоловский,
        В.С. Фоменко}
\institute{
  МИНОБРНАУКИ РОССИИ \\
  Федеральное государственное автономное образовательное учреждение   высшего образования \\
  <<ЮЖНЫЙ ФЕДЕРАЛЬНЫЙ УНИВЕРСИТЕТ>> \\
  Институт Компьютерных технологий и информационной безопасности \\
  Кафедра Математического обеспечения и применения ЭВМ
}
\date{\the\year~год}
\setbeamercovered{transparent}
% Если в нашей презентации используется последовательное высвечивание элементов, стоит сказать: \setbeamercovered{transparent}, чтобы неактивные элементы были хотя бы немного видны.

%\setbeamertemplate{navigation symbols}{}  %убрать панель навигации

%	\author{}
	%\title{}
	%\subtitle{}
	%\logo{}
	%\institute{}
	%\date{}
	%\subject{}
	%\setbeamercovered{transparent} % или dynamic
	%\setbeamertemplate{navigation symbols}{}


\begin{frame}[plain]
	%\maketitle
  \titlepage
\end{frame}

\begin{frame}
  \frametitle{Содержание}
  \tableofcontents
\end{frame}

%%%%%%%%%%%%%%%%%%%%%%%%%%%%%%%%%%%%%%%%%%%%%%%%%%%%%%%%%%%%%%%%%%%%%
\section{Возможности Visual Studio}
% \subsection{Особенности СТУ}

%%%%%%%%%%%%%%%%%%%%%%%%%%  слайд 3 %%%%%%%%%%%%%%%%%%%%%%%%%%%%

\begin{frame}
\frametitle{Качество ПО}

\begin{columns}[c]
\column{0.5\textwidth}

\begin{center}
  \includegraphics[width=0.8\textwidth]{res/img/improve.png}
  Улучшение
\end{center}

\column{0.5\textwidth}
\begin{block}{}
\begin{itemize}
  \item Качество программ -- понятие, аккумулирующее противоречия природы самих программ и противоречия, возникающие при их создании
  \item Профилирование -- сбор характеристик работы программы
  \item Оптимизация -- модификация программы для улучшения её эффективности
\end{itemize}
\end{block}

\end{columns}


\end{frame}




\end{document}
