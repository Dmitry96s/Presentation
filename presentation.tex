% https://habrahabr.ru/post/145523/
% http://fsweb.info/editors/latex/presentation.html
\documentclass[10pt]{beamer}
\usepackage[utf8]{inputenc}
\usepackage[T2A]{fontenc}
\usepackage[english, russian]{babel}
\usepackage{amsmath,amsfonts,amssymb}
\usepackage{graphicx}
%\usetheme{Singapore}
% \usetheme{Berlin}
% \usetheme{CambridgeUS}
\usetheme{CambridgeUS}

%\setbeamercolor{элемент}{bg=цвет1,fg=цвет2}. Здесь «элемент» — название элемента, чей цвет мы хотим изменить (например, «normal text» — обычный текст), «bg» — цвет фона, «fg» — цвет текста.
%\usebackgroundtemplate{\includegraphics[width=\paperwidth,height=\paperheight]{my_backgroung_picture.jpg}}

\addto\captionsrussian{\renewcommand{\figurename}{Рис.}}
\setbeamertemplate{caption}[numbered]

\RequirePackage{caption}
\DeclareCaptionLabelSeparator{defffis}{ -- }
\captionsetup{justification=centering,labelsep=defffis}

\newcommand{\MB}{\mathbf}
\newcounter{myexmpl}



\begin{document}


\title{Эргономика компьютерных игр}
\subtitle{презентация по курсу ``Человеко-компьютерное взаимодействие''}
\logo{\includegraphics[width=0.15\textwidth]{res/img/titlePage/sfedu.png}}
\author{В.Э. Смирнов,
        Д.В. Сухоловский,
        В.С. Фоменко}
\institute{
  МИНОБРНАУКИ РОССИИ \\
  Федеральное государственное автономное образовательное учреждение   высшего образования \\
  <<ЮЖНЫЙ ФЕДЕРАЛЬНЫЙ УНИВЕРСИТЕТ>> \\
  Институт Компьютерных технологий и информационной безопасности \\
  Кафедра Математического обеспечения и применения ЭВМ
}
\date{\the\year~год}
\setbeamercovered{transparent}
% Если в нашей презентации используется последовательное высвечивание элементов, стоит сказать: \setbeamercovered{transparent}, чтобы неактивные элементы были хотя бы немного видны.

%\setbeamertemplate{navigation symbols}{}  %убрать панель навигации

%	\author{}
	%\title{}
	%\subtitle{}
	%\logo{}
	%\institute{}
	%\date{}
	%\subject{}
	%\setbeamercovered{transparent} % или dynamic
	%\setbeamertemplate{navigation symbols}{}


\begin{frame}[plain]
	%\maketitle
  \titlepage
\end{frame}

\begin{frame}
  \frametitle{Содержание}
  \tableofcontents
\end{frame}

%%%%%%%%%%%%%%%%%%%%%%%%%%%%%%%%%%%%%%%%%%%%%%%%%%%%%%%%%%%%%%%%%%%%%
\section{Введение}

% \subsection{Особенности СТУ}

%%%%%%%%%%%%%%%%%%%%%%%%%%  слайд 3 %%%%%%%%%%%%%%%%%%%%%%%%%%%%
\begin{frame}
\frametitle{Игровые приложения}

\begin{block}{Игровые приложения}
  Совершенно особый тип программ. Некоторые игры можно считать своеобразными произведениями искусства, в которых пользователь одновременно выступает в качестве актера и режиссера.
\end{block}

\begin{block}{Разработка интерфейсов игровых программ предполагает}
  \begin{itemize}
    \item обеспечение простоты управления игрой
    \item обеспечение удобства управления игрой
    \item создание у пользователя определенного эмоционального настроя
  \end{itemize}
\end{block}
\end{frame}

\begin{frame}
\frametitle{Эргономика и игры}

\begin{block}{Хорошая игра должна}
  \begin{itemize}
    \item увлекать и всецело затягивать
    \item вызывать чувство эстетического удовлетворения
  \end{itemize}
\end{block}

\begin{block}{Цели}
  \begin{itemize}
    \item \textit{Эргономики} -- адаптировать приложение для использования конкретным человеком в конкретной ситуации
    \item \textit{Компьютерных игр} -- развлечение, получение удовольствия
  \end{itemize}
\end{block}

\begin{block}{}
  \textit{Компьютерные игры} являются весьма специфическим видом ПО, в котором удовлетворение пользователя обусловлено целым рядом специфических факторов.
\end{block}

\end{frame}

\section{Факторы, влияющие на удовлетворение игрока}
\begin{frame}
\frametitle{Факторы, влияющие на удовлетворение игрока}

\begin{block}{}
  К числу наиболее важных для пользователя характеристик игрового ПО относятся следующие:
  \begin{itemize}
    \item Простота использования
    \item Ритм игры
    \item Степень трудности игры
  \end{itemize}
\end{block}
\end{frame}

\subsection{Простота использования}
\begin{frame}
\frametitle{Простота использования}

\begin{block}{}
  Данный критерий принадлежит к числу решающих при оценке пользователями любой компьютерной программы. В случае с играми простота очень тесно связана с легкостью овладения. Не следует забывать и о понятности интерфейса, а также о легкости использования периферийных устройств.
  \begin{itemize}
    \item Начало игры
    \item Обучающие уровни
    \item Игровой интерфейс
    \item Управление игрой с помощью периферийных устройств
  \end{itemize}
\end{block}

\end{frame}

\begin{frame}

\begin{block}{Начало игры}
  Меню, с помощью которого осуществляется запуск игры, следует уделять особое внимание. Разработчик должен быть уверенным в том, что:
  \begin{itemize}
    \item пользователь получил исчерпывающую информацию о возможностях игры
    \item после ознакомления с меню пользователь может управлять игровым процессом без проблем
  \end{itemize}
\end{block}

\begin{block}{Обучающие уровни}
  Если игра включает обучающие уровни, то они должны быть органичной ее частью. Они должны отвечать следующим требованиям:
  \begin{itemize}
    \item не быть как слишком трудными, так и слишком легкими
    \item после прохождения обучающих уровней у пользователя должно возникнуть желание продолжать игру
  \end{itemize}
\end{block}

\end{frame}


\begin{frame}

\begin{block}{Игровой интерфейс}
  Его задачей является обеспечение обратной связи с пользователем, а также осуществление некоторых действий. К игровым интерфейсам применяются такие же требования, как и к интерфейсу любого ПО.
\end{block}

\begin{block}{Управление игрой с помощью периферийных устройств}
  Управление любой периферией должно осуществляться <<на автомате>>. Выбор кнопок и их функции должны быть обусловлены исключительно удобством пользователя.
\end{block}

\end{frame}

\begin{frame}
\frametitle{Рекомендации}

\begin{block}{}
  \begin{itemize}
    \item Избегайте продолжительных вступлений
    \item Старайтесь как можно органичнее встраивать индикаторы в игровое окружение
    \item Используйте модели повреждения или другие уникальные контекстуальные решения вместо индикаторов здоровья
    \item Используйте простую контекстно зависимую систему управления вместо сложной и обширной
  \end{itemize}
\end{block}

\end{frame}

\subsection{Ритм игры}
\begin{frame}
\frametitle{Ритм игры}

\begin{block}{}
  Здесь речь идет обо всем, что способствует <<затягиванию>> пользователя в игровой процесс, и в первую очередь -- об интерактивности и о сюжете игры.
  \begin{itemize}
    \item Интерактивность
    \item Сюжет
  \end{itemize}
\end{block}

\end{frame}

\begin{frame}

\begin{block}{Интерактивность}
  Смысл интерактивности заключается в создании у пользователя уверенности в том, что он управляет всем игровым процессом
\end{block}

\begin{block}{Сюжет}
  Хорошо продуманный сюжет позволяет увеличить степень интеграции пользователя в игровой процесс. Отношения между персонажами игры и пользователем, а также их смысловой контекст являются в данном случае главными факторами.
\end{block}

\end{frame}

\begin{frame}
\frametitle{Рекомендации}

\begin{block}{}
  \begin{itemize}
    \item Больше исследований, меньше объяснений
    \item Избегайте дополнительных предметов коллекций, объясняющих важные моменты из истории вашего игрового мира
    \item Молчание может быть не менее содержательным, чем полноценный диалог
    \item Заставляйте игрока ставить вопросы и не бойтесь оставлять их без ответа
    \item Очень простой сюжет можно компенсировать, выдвинув на первый план сложный процесс совершенствования игрока
  \end{itemize}
\end{block}

\end{frame}

\subsection{Степень трудности игры}
\begin{frame}
\frametitle{Степень трудности игры}

\begin{block}{}
  Степень трудности игры является одним из важных факторов, определяющих ее успешность. Игровой процесс должен быть организован так, чтобы задачи, возникающие перед пользователем, были достаточно сложными и чтобы их сам процесс их решения был приятен, интересен для пользователя. Уровень трудности игры должен соответствовать уровню пользовательской компетенции.
  \begin{itemize}
    \item Сложность
    \item Тип задачи
    \item Возможность компенсации
  \end{itemize}
\end{block}

\end{frame}

\begin{frame}

\begin{block}{Сложность}
  Уровень сложности игры должен быть обусловлен в первую очередь уровнем компетенции пользователя. По мере освоения пользователем специфики игры должны становиться более сложными и решаемые задачи.
\end{block}

\begin{block}{Тип задачи}
  Тип решаемых задач также очень важен. Специфика игровых задач должна соответствовать общему контексту игры и отвечать ожиданиям пользователя.
\end{block}

\begin{block}{Возможность компенсации}
  Зачастую на решение той или иной задачи затрачивается немало времени и сил. Организация игрового процесса должна предусматривать возможность компенсации усилий, направленных на преодоление трудностей.
\end{block}

\end{frame}

\begin{frame}
\frametitle{Рекомендации}

\begin{block}{}
  \begin{itemize}
    \item Всегда отдавайте предпочтение чему-либо знакомому, а не уникальному
    \item Используйте при построении головоломок и испытаний возможности ритма и синхронности
    \item Принуждайте игроков, открывших новую способность, самостоятельно упражняться и экспериментировать с ней
    \item Нет излишествам
  \end{itemize}
\end{block}

\end{frame}

\end{document}
