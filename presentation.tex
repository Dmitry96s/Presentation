\documentclass[10pt]{beamer}
\usetheme{Pittsburgh}%Penguins
\usepackage[utf8]{inputenc}
\usepackage[OT1]{fontenc}
\usepackage[english, russian]{babel}
\usepackage{colortbl}
\setbeamertemplate{caption}[numbered]

%Information to be included in the title page:
\title{Индивидуальная работа по курсу ТПР}
\subtitle{Часть 3}
\author{Сухоловский Дмитрий}
\institute{Южный Федеральный Университет}
\date{2017}



\begin{document}

\frame{\titlepage}

\begin{frame}
\frametitle{Задание}
  \begin{enumerate}
    \item Найти наилучшую альтернативу, используя заданные в варианте методы анализа коллективных решений.
    \item Сравнить полученные результаты, сделать выводы. Сформулировать рекомендации ЛПР по итоговому решению.
  \end{enumerate}
  Исходные данные голосования приведены в таблице \ref{t1}.
\end{frame}

\begin{frame}
\frametitle{Исходные данные}
  \begin{table}
    \caption{Результаты голосования}
    \label{t1}
    \begin{tabular}{|c|c|}
      \hline
      Число голосов & Варианты упорядочивания проектов \\
      \hline
      20	&  A $\succ$ B $\succ$ C $\succ$ D \\
      50	&  A $\succ$ B $\succ$ D $\succ$ C \\
      2	  &  A $\succ$ C $\succ$ B $\succ$ D \\
      3	  &  A $\succ$ C $\succ$ D $\succ$ B \\
      12	&  A $\succ$ D $\succ$ B $\succ$ C \\
      34	&  B $\succ$ A $\succ$ D $\succ$ C \\
      30	&  B $\succ$ C $\succ$ D $\succ$ A \\
      18	&  C $\succ$ A $\succ$ D $\succ$ B \\
      41	&  C $\succ$ D $\succ$ A $\succ$ B \\
      21	&  D $\succ$ A $\succ$ C $\succ$ B \\
      32	&  C $\succ$ B $\succ$ A $\succ$ D \\
      6	  &  D $\succ$ B $\succ$ A $\succ$ C \\
      15	&  D $\succ$ B $\succ$ C $\succ$ A \\
      15	&  D $\succ$ C $\succ$ A $\succ$ B \\
      1	  &  D $\succ$ C $\succ$ B $\succ$ A \\
      \hline
    \end{tabular}
  \end{table}
\end{frame}

\begin{frame}
  \frametitle{Процедура Борда}
\begin{columns}
  \column{0.5\textwidth}
    Для начала рассчитаем попарные предпочтения кандидатов g(A,B), т.е. количество голосовавших, предпочитающих Вариант А варианту B для каждого проекта.
  \column{0.5\textwidth}
  g(A,B) = 182 \quad  g(B,A) = 118\\
  g(A,C) = 148 \quad  g(C,A) = 152\\
  g(A,D) = 171 \quad  g(D,A) = 129\\
  g(B,C) = 167 \quad  g(C,B) = 133\\
  g(B,D) = 168 \quad  g(D,B) = 132\\
  g(C,D) = 146 \quad  g(D,C) = 154\\
\end{columns}
\end{frame}

\begin{frame}
  \frametitle{Процедура Борда}
  \framesubtitle{(продолжение)}
  \begin{columns}
    \column{0.5\textwidth}
    Для каждого проекта рассчитаем значение модифицированной функции Борда по формуле:

    \begin{equation*}
      f_{BM}(A_{i})=\sum\limits_{j}[\,g(A_{i},A_{j})-g(A_{j},A_{i})]\,
    \end{equation*}

    Победителем становится тот, у кого она больше.

    \column{0.5\textwidth}
    Результаты:\\
    А = 102\\
    B = 6\\
    C = -38\\
    D = -70\\
    \textbf{Победителем стал проект А.}

  \end{columns}
\end{frame}

\begin{frame}
  \frametitle{Процедура Фишберна}
  \begin{columns}
  \column{0.5\textwidth}

  \begin{table}
    \caption{Матрица парных сравнений альтернатив}
    \label{t2}
    \begin{tabular}{|c|c|c|c|c|c|}
      \hline
        &\textbf{A} &	\textbf{B} &	\textbf{C} &	\textbf{D} &	\textbf{Sum}\\
        \hline
      \textbf{A} & 0 &	1 &	0 &	1 &	\cellcolor{yellow}2\\
      \textbf{B} & 0 &	0 &	1 &	0 &	1\\
      \textbf{C} & 1 &	0 &	0 &	0 &	1\\
      \textbf{D} & 0 &	1 &	1 &	0 &	\cellcolor{yellow}2\\
      \hline
    \end{tabular}
  \end{table}
  \column{0.5\textwidth}
  Согласно этой процедуре победителем станет тот, кто превзойдет соперников при попарных сравнениях большее число раз.

  Занесем результаты сравнений в таблицу \ref{t2}, элементы которой равны единице, если $g(i,j)>g(j,i)$ и нулю в противном случае.

  \textbf{Победителями стали A и D.}
  \end{columns}
\end{frame}

\begin{frame}
  \frametitle{Процедура Доджсона}
  \begin{columns}
    \column{0.6\textwidth}
    \begin{table}
      \caption{Вспомогательная матрица}
      \label{t3}
      \def\arraystretch{1.5}
      \begin{tabular}{|c|c|c|c|c|}
        \hline
        $H$ &	\textbf{A} &	\textbf{B} &	\textbf{C} &	\textbf{D} \\
        \hline
        \textbf{A} &	$1       $& 	$1  \frac{32}{59}$& 	  $\frac{37}{38}$ &	$ 1 \frac{14}{43}$\\
        \hline
        \textbf{B} &	 $\frac{59}{91}$&	$1$ & $1\frac{34}{133}$ &	$1 \frac{3}{11}$\\
        \hline
        \textbf{C} & $1\frac{1}{37}$ & 	 $\frac{133}{167}$ &	$1$  &	  $\frac{73}{77}$ \\
        \hline
        \textbf{D} &	  $\frac{43}{57}$ &	  $\frac{11}{14}$ 	& $1$   $\frac{4}{73}$ 	& $1$        \\
          \hline
      \end{tabular}
    \end{table}
    \column{0.4\textwidth}
    Для этой процедуры нам понадобится вспомогательная матрица (Таблица \ref{t3}), элементы которой рассчитываются по формуле:

    \begin{equation*}
      h_{i,j} = \frac{g(A_{i}, A_{j})}{g(A_{j}, A_{i})}
    \end{equation*}
  \end{columns}
\end{frame}

\begin{frame}
  \frametitle{Процедура Доджсона}
  \framesubtitle{(продолжение)}
  \begin{columns}
    \column{0.6\textwidth}
    \begin{table}
      \caption{Матрица недостающих голосов}
      \label{t4}
      \begin{tabular}{|c|c|c|c|c|c|}
        \hline
            & \textbf{A}  & \textbf{B}  & \textbf{C} & \textbf{D} & \textbf{Sum} \\
            \hline
          \textbf{A}	& 0	 & 0	& 2	& 0	& \cellcolor{yellow}2 \\
          \textbf{B}	& 32 & 0	& 0	& 0	& 32 \\
          \textbf{C}	& 0	 & 17	& 0	& 4	& 21 \\
          \textbf{D}	& 21 & 18	& 0	& 0	& 39 \\
          \hline
      \end{tabular}
    \end{table}
    Для каждого варианта $A_{i}$ по строке матрицы $H$ находятся варианты $A_{j}$ для которых $H_{i,j} < 1$. Это как раз и будут те варианты, в сравнении с которыми вариант $A_{i}$ получил меньше половины от общего числа голосов.
    \column{0.4\textwidth}

Подсчитаем недостающие голоса и занесем в таблицу \ref{t4}.

Победителем признан тот, у кого значение функции меньше - проект А.

\textbf{Ранжирование: $A \succ C \succ B \succ D$}

  \end{columns}
\end{frame}

\begin{frame}
  \frametitle{Процедура Кумбса}
  \begin{columns}
  \column{0.5\textwidth}
  \begin{table}
    \caption{Процедура Кумбса}
    \label{t5}
    \begin{tabular}{|c|c|c|c|c|}
      \hline
      \textbf{Шаг} & \textbf{A} &\textbf{B} &\textbf{C} &\textbf{D}\\
      \hline
      \textbf{1} &	46  & 98  & \cellcolor{yellow}102 & 54\\
      \textbf{2} &	52  & 110 & -   & \cellcolor{yellow}138\\
      \textbf{3} &	118 & \cellcolor{yellow}182 &     &\\
      \hline
    \end{tabular}
  \end{table}

  \column{0.5\textwidth}
  На каждой итерации алгоритма, будем подсчитывать число голосовавших, поставивших каждый вариант на последнее место.

Вариант против которого проголосовало большинство будем исключать.

Будем продолжать до тех пор, пока не останется победитель.

Результаты расчетов находятся в таблице \ref{t5}.

\textbf{Победитель  А}
\end{columns}
\end{frame}

\begin{frame}
  \frametitle{Результаты}
  \begin{table}
    \caption{Результаты процедур}
    \label{t6}
    \begin{tabular}{|c|c|}
      \hline
      \textbf{Процедура} & \textbf{Результат}\\
      \hline
      Модифицированная процедура Борда & $A \succ B \succ C \succ D$\\
      Процедура Кондорсе (Нет решения)	&   $C \succ A \succ B \succ D \succ C$ \\
      Процедура Кумбса	  & $A \succ B \succ D \succ C$\\
      Процедура Доджсона	& $A \succ C \succ B \succ D$\\
      Процедура Фишберна	& $A = D \succ B =C $\\
      \hline
    \end{tabular}
  \end{table}
\end{frame}

\begin{frame}
  \frametitle{Вывод}
  \textbf{ЛПР следует выбрать проект А, так как он победил по всем процедурам кроме Процедуры Кондорсе, которая не дала решения, из-за нарушения транзитивности.}

\end{frame}

\end{document}
