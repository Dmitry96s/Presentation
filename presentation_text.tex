\documentclass{article}


\usepackage[14pt]{extsizes}
\usepackage[top=20mm, bottom=20mm, left=30mm, right=10mm, paperwidth=210mm, paperheight=297mm]{geometry}
\usepackage{amsmath,amsthm,amssymb}
\usepackage{mathtext}
\usepackage{cmap}
\usepackage[T1,T2A]{fontenc}
\usepackage[utf8]{inputenc}
\usepackage[russian]{babel}
\usepackage{graphicx,subcaption,caption,float}
\usepackage{fancyhdr}
\usepackage{listings}
\usepackage{multirow}
\usepackage{color}
\usepackage{indentfirst}
\usepackage{tabulary}

%% Setting up packages

% listings
\definecolor{colorSourceComment}{rgb}{0,0.6,0}
\definecolor{colorSourceString}{rgb}{0.58,0,0.82}

\lstset{
  basicstyle=\footnotesize\sffamily,
  breaklines=true,
  captionpos=t,
  commentstyle=\color{colorSourceComment},
  frame=l,
  keepspaces=true,
  keywordstyle=\color{blue},
  numbers=left,
  numbersep=5pt,
  numberstyle=\footnotesize\sffamily,
  % showstringspaces=false, %%
  stringstyle=\color{colorSourceString},
  tabsize=2,
  title=\lstname
}

% fancyhdr
\renewcommand{\headrulewidth}{0pt}

\newcommand{\Section}[1]
{
  \newpage
  \section{#1}
}

\newcommand{\Sectionx}[1]
{
  \newpage
  \section*{#1}
  \addcontentsline{toc}{section}{#1}
}

\newcommand{\ssection}[1]{\subsection{#1}}
\newcommand{\sssection}[1]{\subsubsection{#1}}
\renewcommand\labelitemi{---}

%% Setting up document info
\title{Test}
\author{RubyScorp}
\date{\today}

\begin{document}
\section{Введение}
Игровые приложения представляют собой совершенно особый тип программ. Некоторые игры можно считать своеобразными произведениями искусства, в которых пользователь одновременно выступает в качестве актера и режиссера. Разработка интерфейсов игровых программ предполагает не только решение сугубо утилитарных задач, связанных с обеспечением простоты и удобства управления игрой, но еще и создание у пользователя определенного эмоционального настроя.

\bigskip\noindent Хорошая игра должна,
\begin{itemize}
  \item во-первых, \text{увлекать и всецело затягивать},
  \item во-вторых -- \text{вызывать чувство эстетического удовлетворения}.
\end{itemize}

\textbf{Цель -- удовлетворение пользователя}

Компьютерная эргономика и эргономика компьютерных игр имеют одинаковую цель: адаптировать приложение для использования конкретным человеком в конкретной ситуации. Для того, чтобы сделать приложение действительно удобным в использовании, необходимо знать, для каких целей оно будет применяться. Если речь идет о компьютерных играх, то их целью является развлечение, получение удовольствия.

Задачей эргономики является создание оптимальных условий труда, создающих у человека чувство удовлетворения от проделанной работы. При работе с программным обеспечением чувство удовлетворения возникает, если пользователь успешно и без лишних усилий выполняет с помощью программы ту или иную задачу. Компьютерные игры являются весьма специфическим видом ПО, в котором удовлетворение пользователя обусловлено целым рядом специфических факторов.

\section{Факторы, влияющие на удовлетворение пользователя игрового ПО}
Анализируя особенности юзабилити компьютерных игр, необходимо принимать во внимание специфику игрового опыта. К числу наиболее важных для пользователя характеристик игрового ПО относятся следующие:

\subsection{Простота использования}
Данный критерий принадлежит к числу решающих при оценке пользователями любой компьютерной программы. В случае с играми простота очень тесно связана с легкостью овладения. Не следует забывать и о понятности интерфейса, а также о легкости использования периферийных устройств.

\textbf{Начало игры} -- Меню, с помощью которого осуществляется запуск игры, следует уделять особое внимание. Разработчик должен быть уверенным в том, что:
\begin{itemize}
  \item пользователь получил исчерпывающую информацию о возможностях игры;
  \item после ознакомления с меню пользователь может управлять игровым процессом без проблем
\end{itemize}

\textbf{Обучающие уровни} -- Если игра включает обучающие уровни, то они должны быть органичной ее частью. Они должны отвечать следующим требованиям:
\begin{itemize}
  \item не быть как слишком трудными, так и слишком легкими;
  \item после прохождения обучающих уровней у пользователя должно возникнуть желание продолжать игру
\end{itemize}

\textbf{Игровой интерфейс} -- Его задачей является обеспечение обратной связи с пользователем, а также осуществление некоторых действий. К игровым интерфейсам применяются такие же требования, как и к интерфейсу любого ПО.

\textbf{Управление игрой с помощью периферийных устройств} -- Управление любой периферией должно осуществляться <<на автомате>>. Выбор кнопок и их функции должны быть обусловлены исключительно удобством пользователя.

\bigskip

Элементы, не относящиеся собственно к игре, не должны отвлекать пользователя от игрового процесса. Меню, сохранение уровней и т. п. -- все это должно быть предельно простым и не занимать большую часть внимания пользователя. Впрочем, в некоторых случаях бывает необходимо руководствоваться не столько критериями удобства, сколько эстетики. Визуально привлекательный интерфейс, анимация, выполняющая исключительно декоративные функции, необычное оформление меню -- все это может способствовать погружению, затягиванию пользователя в игру и тем самым повышать степень его удовлетворенности.

\section{Ритм игры}
Здесь речь идет обо всем, что способствует «затягиванию» пользователя в игровой процесс, и в первую очередь — об интерактивности и о сюжете игры.

\textbf{Интерактивность} -- Смысл интерактивности заключается в создании у пользователя уверенности в том, что он управляет всем игровым процессом

\textbf{Сюжет} -- Хорошо продуманный сюжет позволяет увеличить степень интеграции пользователя в игровой процесс. Отношения между персонажами игры и пользователем, а также их смысловой контекст являются в данном случае главными факторами.

\section{Степень трудности игры}
Степень трудности игры является одним из важных факторов, определяющих ее успешность. Игровой процесс должен быть организован так, чтобы задачи, возникающие перед пользователем, были достаточно сложными и чтобы их сам процесс их решения был приятен, интересен для пользователя. Уровень трудности игры должен соответствовать уровню пользовательской компетенции.

\textbf{Сложность} -- Уровень сложности игры должен быть обусловлен в первую очередь уровнем компетенции пользователя. По мере освоения пользователем специфики игры должны становиться более сложными и решаемые задачи.

\textbf{Тип задачи} -- Тип решаемых задач также очень важен: так, в автомобильных гонках пользователь должен не угадывать, в какую сторону повернет автомобиль после нажатия той или иной кнопки, но научиться управлять автомобилем. Специфика игровых задач должна соответствовать общему контексту игры и отвечать ожиданиям пользователя.

\textbf{Возможность компенсации} -- Пользователь игрового приложения в отличие от пользователя офисного ПО является мотивированным исследователем. Зачастую на решение той или иной задачи затрачивается немало времени и сил. Организация игрового процесса должна предусматривать возможность компенсации усилий, направленных на преодоление трудностей.

\bigskip

Перечисленные выше критерии основаны на рекомендациях группы по тестированию компьютерных игр компании Microsoft. Учет этих критериев позволяет существенно улучшить игру и максимально приблизить ее к ожиданиям пользователя.

\end{document}
